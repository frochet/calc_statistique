\documentclass{article}

\usepackage[utf8]{inputenc}
\usepackage{lmodern}
\usepackage{amsmath}
\usepackage{graphicx}
\usepackage{url}

\usepackage{alltt}
\usepackage{verbatim}

\usepackage{listings}
\usepackage{color}
\usepackage[    colorlinks,%
                citecolor=black,%
                filecolor=black,%
                linkcolor=black,%
                urlcolor=black  ]{hyperref}
\usepackage{attachfile}

\usepackage{layout}
\usepackage[top=3cm, bottom=3cm, left=3cm, right=3cm]{geometry}

\usepackage{enumitem}
\usepackage{listings}
\usepackage{caption}
\usepackage{subcaption}

\usepackage{color}
\definecolor{gris}{rgb}{0.97,0.97,0.97}
\definecolor{mymauve}{rgb}{0.58,0,0.82}
\definecolor{mygreen}{rgb}{0,0.6,0}

\lstset{numbers=left, tabsize=4, backgroundcolor=\color{gris},
frame=single, breaklines=true,
keywordstyle=\color{blue},
commentstyle=\color{mygreen},    % comment style
numberstyle=\footnotesize\color{black}, % the style that is used for the line-numbers
stringstyle=\color{mymauve},     % string literal style
stringstyle=\ttfamily,
framexleftmargin=6mm, xleftmargin=6mm,
language=C}

% for phasor representation
\usepackage{tikz}
\usetikzlibrary{arrows}
\newdimen\xorig
\newdimen\yorig

% conditions
\usepackage{ifthen}
\newcommand\horizontalkeyword[1]{%
  \pgfmathparse{
    ifthenelse(#1 > 0, "right", "left")}%
  \pgfmathresult
}
\newcommand\verticalkeyword[1]{
\pgfmathsetmacro{\verticalkw}{#1}
\pgfmathparse{ifthenelse(\verticalkw>=0,"above","under")} \pgfmathresult}%

\usepackage{xspace}

\newcommand{\expphasorpi}[6]% origin, r, phi(pi), style, text, textposition
{   \path (#1);
  \pgfgetlastxy{\xorig}{\yorig}
  \pgfmathsetmacro{\xfirst}{\xorig/28.453}
  \pgfmathsetmacro{\yfirst}{\yorig/28.453}
  \pgfmathsetmacro{\xsecond}{\xorig/28.453+#2*cos(#3)}
  \pgfmathsetmacro{\ysecond}{\yorig/28.453+#2*sin(#3)}
  \pgfmathtruncatemacro\inthori{\xsecond}
  \pgfmathtruncatemacro\intvert{\ysecond}
  \draw[#4] (\xfirst,\yfirst) -- (\xsecond,\ysecond) node[
  #6
  ]{#5};
}

\makeatletter
\def\clap#1{\hbox to 0pt{\hss #1\hss}}%
\def\ligne#1{%
\hbox to \hsize{%
\vbox{\centering #1}}}%
\def\haut#1#2#3{%
\hbox to \hsize{%
\rlap{\vtop{\raggedright #1}}%
\hss
\clap{\vtop{\centering #2}}%
\hss
\llap{\vtop{\raggedleft #3}}}}%
\def\bas#1#2#3{%
\hbox to \hsize{%
\rlap{\vbox{\raggedright #1}}%
\hss
\clap{\vbox{\centering #2}}%
\hss
\llap{\vbox{\raggedleft #3}}}}%
\def\maketitle{%
\thispagestyle{empty}\vbox to \vsize{%
\haut{}{\@blurb}{}
\vfill
\vspace{1cm}
\begin{flushleft}
\huge \@title
\end{flushleft}
\par
\hrule height 4pt
\par
\begin{flushright}
\Large \@author
\par
\end{flushright}
\vspace{1cm}
\vfill
\vfill
\bas{}{\@location, on \@date}{}
}%
\cleardoublepage
}
\def\date#1{\def\@date{#1}}
\def\author#1{\def\@author{#1}}
\def\title#1{\def\@title{#1}}
\def\location#1{\def\@location{#1}}
\def\blurb#1{\def\@blurb{#1}}
\date{\today}
\makeatother

\lstset{tabsize=4,
        basicstyle=\scriptsize,
        %upquote=true,
        aboveskip={1.5\baselineskip},
        columns=fixed,
        showstringspaces=false,
        extendedchars=true,
        breaklines=true,
        prebreak = \raisebox{0ex}[0ex][0ex]{\ensuremath{\hookleftarrow}},
		frame=single,
		numbers=left,
		numberstyle=\tiny\color{black},
        showtabs=false,
        showspaces=false,
        showstringspaces=false,
        identifierstyle=\ttfamily,
        keywordstyle=\color[rgb]{0.5,0,0.35}\bfseries,
        commentstyle=\color[rgb]{0.25,0.5,0.35},
        stringstyle=\color[rgb]{0.6,0,0},
        morecomment=[s][\color[rgb]{0.25,0.35,0.75}]{/**}{*/},
	language=Python
}

%opening
\title{LINGI2525: Introduction aux circuits électriques et électroniques \\
Travail 4: Filtre}
\author{Debroux Léonard} 
\date{Année académique 2013-2014}

	
% \title{Report on Assignment 1}
% \author{\begin{tabular}{l p{2cm}} \\
% 	Léonard \textsc{Debroux} \\
% 	Kevin \textsc{Jadin} 
%     \vspace{3mm} \\
% \end{tabular} \\
% }
% \location{Louvain-La-Neuve}
% \blurb{%
% Université Catholique de Louvain\\
% \'Ecole Polytechnique de Louvain\\
% \textbf{LINGI 2143 : Concurrent System}\\[1em]
% }% 

\newcommand{\phasor}[2] {
  #1\phase{#2\degree}
}

\begin{document}

\begin{titlepage}
    \begin{center}
        {\huge LSTAT2020: Calcul statistique sur ordinateur}\\
        \vspace{0.4cm}
        
        {\Large {Professeur : Celine Bugli}}\\
        \vspace{0.6cm}
        
        {\Large \textit{Projet R}}\\
        \vspace{1.2cm}

        \texttt{}\\
        \vspace{0.2cm}

        \includegraphics[height=10cm]{pageGarde.png}\\
        \vspace{0.1cm}
        {\Large \textbf{Universit\'e Catholique de Louvain}}
        \vspace{0.7cm}

        \vspace{2cm}
        
        Florentin Rochet\\
        Léonard Debroux \\
        \vspace{0.2cm}
        2013-2014\\
    \end{center}
\end{titlepage}

Le programme se trouve dans le fichier \texttt{projet.r}. Le fichier est séparé en 3 parties, qui regroupent les déclarations de fonctions utilisées pour chaque partie. A la fin du fichier, vous trouverez les appels de fonctions produisant les résultats attendus pour chaque partie (le changement de répertoire se trouve en fin de fichier).\\
L'exécution prend une à quelques dizaines de secondes selon les confifurations.

\section{Partie A}

\subsection{Remarque}
Concernant le point A.1 ``Modifier le répertoire courant de votre session R afin que celui-ci corresponde au répertoire créé préalablement''. La ligne de code concernée est commentée car nous travaillons en console et donc lançons le script depuis le bon répertoire.\\
Pour lancer le script, il faut donc décommenter \texttt{stwd("répertoire du script")} et mettre le bon chemin pour le répertoire dans lequel le script se trouve. L'autre solution de lancer le script en console à partir du bon répertoire évite d'avoir à changer le chemin.

\subsection{Explication du code}
La fonction principale générant les fichiers demandés se nomme \texttt{compute\_stats()}. Pour chaque province dans le fichier de données, la fonction crée un fichier \texttt{.txt} du nom de la province et y écrit les résultats des fonctions \texttt{compute\_cens()}, \texttt{compute\_trt()} et \texttt{stat\_descr()} dans celui-ci. \\
Le 5ième objectif de la partie A est réalisé à l'aide de la fonction \texttt{standardization()} qui sera appelée dans la suite du code quand l'opération sera demandée.

\section{Partie B}
Les deux fonctions appelées sont \texttt{plot\_graphs()} et \texttt{plot\_function()}. Elles se chargent repectivement des questions 1 et 2.\\
Voici les résultats produits. Vous pouvez obtenir ces figures en executant le script.

\begin{figure}[!h]
   \begin{minipage}[c]{.46\linewidth}
	\includegraphics[scale=0.3]{../code/Flandre_posterieure.jpeg}
	\caption{Partie B.1 - Flandre}
   \end{minipage} \hfill
   \begin{minipage}[c]{.46\linewidth}
	\includegraphics[scale=0.3]{../code/Flandre_survie.jpeg}
	\caption{Partie B.2 - Flandre}
   \end{minipage}
\end{figure}

\section{Partie C}
La fonction \texttt{post\_dist\_log()} assure le calcul de la fonction $\pi$ de la question 1. Le fonction \texttt{metropolis()} constitue l'implémentation de l'algorithme de Metropolis avec \texttt{metropolis\_core()} qui se charge des étapes étant répétées sur les différents paramètres.\\
La fonction \texttt{compute\_metro()} reprend le travail à faire sur un jeu de données afin de fournir les graphes demandés.\\

Ces figures ont été générées par les mêmes fonctions que celle utilisées à la partie B, après que les vecteurs $\lambda$, $\beta_{1}$ et $\beta_{2}$ aient été calculés.

\begin{figure}[!h]
   \begin{minipage}[c]{.46\linewidth}
	\includegraphics[scale=0.3]{../code/Brabant_Wallon_posterieure.jpeg}
	\caption{Partie C - Brabant Wallon}
   \end{minipage} \hfill
   \begin{minipage}[c]{.46\linewidth}
	\includegraphics[scale=0.3]{../code/Brabant_Wallon_survie.jpeg}
	\caption{Partie C - Brabant Wallon }
   \end{minipage}
\end{figure}

\begin{figure}[!h]
   \begin{minipage}[c]{.46\linewidth}
	\includegraphics[scale=0.3]{../code/Bruxelles_posterieure.jpeg}
	\caption{Partie C - Bruxelles}
   \end{minipage} \hfill
   \begin{minipage}[c]{.46\linewidth}
	\includegraphics[scale=0.3]{../code/Bruxelles_survie.jpeg}
	\caption{Partie C - Bruxelles}
   \end{minipage}
\end{figure}
\end{document}

