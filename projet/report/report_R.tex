\documentclass{article}

\usepackage[utf8]{inputenc}
\usepackage{lmodern}
\usepackage{amsmath}
\usepackage{graphicx}
\usepackage{url}

\usepackage{alltt}
\usepackage{verbatim}

\usepackage{listings}
\usepackage{color}
\usepackage[    colorlinks,%
                citecolor=black,%
                filecolor=black,%
                linkcolor=black,%
                urlcolor=black  ]{hyperref}
\usepackage{attachfile}

\usepackage{layout}
\usepackage[top=3cm, bottom=3cm, left=3cm, right=3cm]{geometry}

\usepackage{enumitem}
\usepackage{listings}
\usepackage{caption}
\usepackage{subcaption}

\usepackage{color}
\definecolor{gris}{rgb}{0.97,0.97,0.97}
\definecolor{mymauve}{rgb}{0.58,0,0.82}
\definecolor{mygreen}{rgb}{0,0.6,0}

\lstset{numbers=left, tabsize=4, backgroundcolor=\color{gris},
frame=single, breaklines=true,
keywordstyle=\color{blue},
commentstyle=\color{mygreen},    % comment style
numberstyle=\footnotesize\color{black}, % the style that is used for the line-numbers
stringstyle=\color{mymauve},     % string literal style
stringstyle=\ttfamily,
framexleftmargin=6mm, xleftmargin=6mm,
language=C}

% for phasor representation
\usepackage{tikz}
\usetikzlibrary{arrows}
\newdimen\xorig
\newdimen\yorig

% conditions
\usepackage{ifthen}
\newcommand\horizontalkeyword[1]{%
  \pgfmathparse{
    ifthenelse(#1 > 0, "right", "left")}%
  \pgfmathresult
}
\newcommand\verticalkeyword[1]{
\pgfmathsetmacro{\verticalkw}{#1}
\pgfmathparse{ifthenelse(\verticalkw>=0,"above","under")} \pgfmathresult}%

\usepackage{xspace}

\newcommand{\expphasorpi}[6]% origin, r, phi(pi), style, text, textposition
{   \path (#1);
  \pgfgetlastxy{\xorig}{\yorig}
  \pgfmathsetmacro{\xfirst}{\xorig/28.453}
  \pgfmathsetmacro{\yfirst}{\yorig/28.453}
  \pgfmathsetmacro{\xsecond}{\xorig/28.453+#2*cos(#3)}
  \pgfmathsetmacro{\ysecond}{\yorig/28.453+#2*sin(#3)}
  \pgfmathtruncatemacro\inthori{\xsecond}
  \pgfmathtruncatemacro\intvert{\ysecond}
  \draw[#4] (\xfirst,\yfirst) -- (\xsecond,\ysecond) node[
  #6
  ]{#5};
}

\makeatletter
\def\clap#1{\hbox to 0pt{\hss #1\hss}}%
\def\ligne#1{%
\hbox to \hsize{%
\vbox{\centering #1}}}%
\def\haut#1#2#3{%
\hbox to \hsize{%
\rlap{\vtop{\raggedright #1}}%
\hss
\clap{\vtop{\centering #2}}%
\hss
\llap{\vtop{\raggedleft #3}}}}%
\def\bas#1#2#3{%
\hbox to \hsize{%
\rlap{\vbox{\raggedright #1}}%
\hss
\clap{\vbox{\centering #2}}%
\hss
\llap{\vbox{\raggedleft #3}}}}%
\def\maketitle{%
\thispagestyle{empty}\vbox to \vsize{%
\haut{}{\@blurb}{}
\vfill
\vspace{1cm}
\begin{flushleft}
\huge \@title
\end{flushleft}
\par
\hrule height 4pt
\par
\begin{flushright}
\Large \@author
\par
\end{flushright}
\vspace{1cm}
\vfill
\vfill
\bas{}{\@location, on \@date}{}
}%
\cleardoublepage
}
\def\date#1{\def\@date{#1}}
\def\author#1{\def\@author{#1}}
\def\title#1{\def\@title{#1}}
\def\location#1{\def\@location{#1}}
\def\blurb#1{\def\@blurb{#1}}
\date{\today}
\makeatother

\lstset{tabsize=4,
        basicstyle=\scriptsize,
        %upquote=true,
        aboveskip={1.5\baselineskip},
        columns=fixed,
        showstringspaces=false,
        extendedchars=true,
        breaklines=true,
        prebreak = \raisebox{0ex}[0ex][0ex]{\ensuremath{\hookleftarrow}},
		frame=single,
		numbers=left,
		numberstyle=\tiny\color{black},
        showtabs=false,
        showspaces=false,
        showstringspaces=false,
        identifierstyle=\ttfamily,
        keywordstyle=\color[rgb]{0.5,0,0.35}\bfseries,
        commentstyle=\color[rgb]{0.25,0.5,0.35},
        stringstyle=\color[rgb]{0.6,0,0},
        morecomment=[s][\color[rgb]{0.25,0.35,0.75}]{/**}{*/},
	language=Python
}

%opening
\title{LINGI2525: Introduction aux circuits électriques et électroniques \\
Travail 4: Filtre}
\author{Debroux Léonard} 
\date{Année académique 2013-2014}

	
% \title{Report on Assignment 1}
% \author{\begin{tabular}{l p{2cm}} \\
% 	Léonard \textsc{Debroux} \\
% 	Kevin \textsc{Jadin} 
%     \vspace{3mm} \\
% \end{tabular} \\
% }
% \location{Louvain-La-Neuve}
% \blurb{%
% Université Catholique de Louvain\\
% \'Ecole Polytechnique de Louvain\\
% \textbf{LINGI 2143 : Concurrent System}\\[1em]
% }% 

\newcommand{\phasor}[2] {
  #1\phase{#2\degree}
}

\begin{document}

\begin{titlepage}
    \begin{center}
        {\huge LSTAT2020: Calcul statistique sur ordinateur}\\
        \vspace{0.4cm}
        
        {\Large {Professeur : Celine Bugli}}\\
        \vspace{0.6cm}
        
        {\Large \textit{Projet R}}\\
        \vspace{1.2cm}

        \texttt{}\\
        \vspace{0.2cm}

        \includegraphics[height=10cm]{pageGarde.png}\\
        \vspace{0.1cm}
        {\Large \textbf{Universit\'e Catholique de Louvain}}
        \vspace{0.7cm}

        \vspace{2cm}
        
        Florentin Rochet\\
        Léonard Debroux \\
        \vspace{0.2cm}
        2013-2014\\
    \end{center}
\end{titlepage}

Le programme se trouve dans le fichier projet.r. Le fichier est séparé en 3 parties, qui regroupent les déclarations de fonctions utilisées pour chaque partie. A la fin du fichier, vous trouverez les appels de fonctions produisant les résultats attendus pour chaque partie.

\section{Partie A}
\subsection{Remarque}
Concernant la première consigne du projet, c'est à dire "modifier le répertoire courant de votre session R afin que celui-ci corresponde au répertoire créé préalablement". Il est en fait plus simple, et surtout portable, d'exécuter un script R dans le répertoire où se trouve le fichier. Sinon vous êtes obligés de changer cette ligne quand vous l'exécuterai. Les étudiants ayant ajouté un setwd("path\_to\_file") ont été obligé de modifié cette ligne a chaque échange de fichier, ce qui peut être légèrement agaçant.\\
Pour cette raison, nous n'utilisons pas setwd(). Nous avons néanmoins laissé un setwd en commentaire au début du fichier si vous préférez utiliser cette méthode.
\subsection{Explication du code}

La fonction principale générant les fichiers demandés se nomme $compute\_stats$. Pour chaque province dans le fichier de données, cette fonction crée un fichier .txt du nom de la province et print les résultats des fonctions $compute\_cens$, $compute\_trt$ et $stat\_descr$ dans celui-ci. \\
Le 5 ème point de cette partie se trouve dans la fonction $standardization$

\section{Partie B}
Voici les résultats produits. Vous pouvez obtenir ces figures en executant le script.

\begin{figure}[!h]
   \begin{minipage}[c]{.46\linewidth}
	\includegraphics[scale=0.3]{../code/Flandre_posterieur.jpeg}
	\caption{Partie B.1 - Braba}
   \end{minipage} \hfill
   \begin{minipage}[c]{.46\linewidth}
	\includegraphics[scale0.3]{../code/Flandre_survie.jpeg}
	\caption{Partie B.scale1 - Flandre}
   \end{minipage}
\end{figure}

\begin{figure}[!h]
   \begin{minipage}[c]{.46\linewidth}
	\includegraphics[scale=0.3]{../code/Flandre_posterieur.jpeg}
	\caption{Partie B.1 - Flandre}
   \end{minipage} \hfill
   \begin{minipage}[c]{.46\linewidth}
	\includegraphics[scale0.3]{../code/Flandre_survie.jpeg}
	\caption{Partie B.2 - Flandre}
   \end{minipage}
\end{figure}

\section{Partie C}


\end{document}

